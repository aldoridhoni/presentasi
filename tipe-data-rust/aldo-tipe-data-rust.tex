\documentclass[14pt]{beamer}
\hypersetup{pdfstartview={Fit}}
\usetheme{metropolis}
\usepackage[utf8]{inputenc}
\setmonofont{DejaVu Sans Mono}
\usepackage{graphicx}
\usepackage{mathtools}
\usepackage{diagbox}
\usepackage{color, colortbl}
\usepackage{minted}
\usepackage{upquote}
\usemintedstyle{friendly}
\definecolor{bg}{rgb}{0.95,0.95,0.95}
\setminted[rust]{
    autogobble,
    fontsize=\small,
    framesep=2mm,
%    bgcolor=bg,
%    fontfamily=tt,
    breaklines=true}

%%% Background image in upper right corner.
\usebackgroundtemplate%
{%
    \tikz[overlay,remember picture] %
    \node[opacity=0.05, yshift=-3cm, at=(current page.45)] {
        \includegraphics[width=0.4\paperwidth]{rust-logo-512x512-blk}};
}

%%% Custom Command for highlight
\newcommand{\hilite}[1] {%
\texttt{\alert{#1}}}

\begin{document}
\author{Muhamad Aldo Ridhoni}
\title{Tipe Data Rust}
\subtitle{Primitive data types}
% \logo{}
\institute{RUST ID Meetup \#2}
\date{Padepokan ASA Wedomartani, 24 September 2017}
% \subject{}
% \setbeamercovered{transparent}
% \setbeamertemplate{navigation symbols}{}
{
\usebackgroundtemplate%
{%
    \tikz[overlay,remember picture] %
    \node[opacity=0.1, xshift=-2.7cm, yshift=2cm, at=(current page.330)] { \includegraphics[width=0.6\paperwidth]{rust-logo-512x512}};}%
\begin{frame}[plain]
  \maketitle
  \addtocounter{framenumber}{-1}
\end{frame}
}

\begin{frame}{Membahas}
  \begin{itemize}[<+->]
  \item Tipe Data Dasar
  \item Penggunaan
  \item Konversi
  \end{itemize}
\end{frame}

\begin{frame}{Tipe Data Dasar}
  \textbf{Scalar} \emph{single value} : 
  \begin{itemize}
  \item Boolean
  \item Integer dan floating-point
  \item Character
  \end{itemize}
  \pause
  \textbf{Compound} \emph{multiple value} :
  \begin{itemize}
  \item Tuples
  \item Arrays
  \item Slices
  \item String
%  \item Pointer
  \end{itemize}
\end{frame}

\begin{frame}{Boolean}
Tipe boolean dengan nilai \textbf{true} atau \textbf{false}.
\end{frame}

\begin{frame}[fragile]{Boolean}
\begin{minted}{rust}
        let benar: bool = true;
        let salah: bool = false;
        
        if benar {
            println!("Berhasil")
        }
        
        assert_eq!(benar as i8, 1);
        assert_eq!(salah as i8, 0);
\end{minted}
\end{frame}

\begin{frame}{Integer}
Integer adalah tipe data yang merepresentasikan bilangan bulat seperti dalam matematika.\\
\begin{description}
    \item[Signed] Cakupan integer bertanda dari $-\left ( 2^{n-1} \right )$ hingga $2^{n-1}-1$.\\
    \item[Unsigned] Integer tidak bertanda dari \\$0$ hingga $2^{n}-1$.
\end{description}

\end{frame}

\begin{frame}{Integer}
\begin{columns}[c]
    \column{.4\textwidth}
    {\footnotesize
    \begin{tabular}{|c|c|c|}
        \hline
        \rowcolor{black!20}Panjang & \emph{Signed} & \emph{Unsigned}\\
        \hline
        8-bit & i8  & u8\\
        16-bit & i16 & u16\\ 
        32-bit & i32 & u32\\
        64-bit & i64 & i64\\
        arch & isize & usize\\
        \hline
    \end{tabular}
    }
    \column{.6\textwidth}
    {\footnotesize
    Cakupan
    \begin{itemize}
        \item \textbf{i8} = -128 .. 127\\
        \textbf{u8} = 0 .. 255
        \item \textbf{i16} = −32.768 .. 32.767\\
        \textbf{u16} = 0 .. 65.535
        \item \textbf{i32} = −2.147.483.648 .. 2.147.483.647\\
        \textbf{u32} = 0 .. 4.294.967.295
        \item \textbf{i64} = −9.223.372.036.854.775.808 .. 9.223.372.036.854.775.807\\
        \textbf{u64} = 0 .. 18.446.744.073.709.551.615
    \end{itemize}
    }
\end{columns}
\end{frame}

\begin{frame}{Integer Literals}
\centering
\begin{tabular}{|c|c|}
    \hline
    \rowcolor{black!20}Jenis & Contoh\\
    \hline
    Desimal & \texttt{45\_678}\\
    Hexadesimal & \texttt{0xbb}\\ 
    Oktal & \texttt{0o77}\\
    Binary & \texttt{0b1100\_1010}\\
    Byte (u8) & \texttt{b'Z'}\\
    \hline
\end{tabular}
\end{frame}
\begin{frame}{Floating-Point}
Floating-point adalah tipe numerik dengan titik desimal.
\begin{description}
    \item [f32] 32bit single precision float.\\
    32bit dengan presisi 24bit.
    \item [f64] 64bit double precision float.\\
    64bit dengan presisi 53bit.
\end{description}

Tipe default untuk float adalah \hilite{f64}.
\end{frame}

\begin{frame}[fragile]{Floating-Point}

\begin{minted}{rust}
let f1: f32 = 1.1234567;
let f2: f32 = 0.00000006;
let f3: f32 = 1.1234568;

assert_eq!(f1 + f2, f3); // true
\end{minted}
\end{frame}

\begin{frame}[fragile]{Character}
Char adalah tipe data yang memuat nilai \hilite{Unicode Scalar}. Satu karakter dimuat dalam petik \hilite{''}.
\end{frame}

\begin{frame}[fragile]{Character}
\begin{minted}{rust}
let c = 'z';
let z = 'ℤ';
let black_chess_knight = '♞';

// hanya u8
let y: char = char::from(0x79);
let x: char = char::from(b'x');
\end{minted}
\end{frame}
\note{
    %use std::str::FromStr;
    %let x = i32::from_str("x");
}

\begin{frame}[fragile]{Tuple}
Tuple \hilite{(T, U, ..)} adalah rangkaian yang dapat terdiri dari elemen dengan tipe berbeda dan dimuat dalam tanda kurung \hilite{()}.

Elemen tuple dapat diakses dengan indeks '\textit{tuple indexing}'.
\end{frame}

\begin{frame}[fragile]{Tuple}
\begin{minted}{rust}
type Pair<T1, T2> = (T1, T2);
let p: Pair<i8, f32> = (10, 3.14);
let q: Pair<&str, i8> = ("ID", 9);
let (a, b) = p;

assert_eq!(a, 10);
assert_eq!(b, 3.14);
assert_eq!(p.0, 10);
assert_eq!(p.1, 3.14);
\end{minted}
\end{frame}

\begin{frame}{Array}
Array \hilite{([T; N])} adalah koleksi padat elemen-elemen dengan tipe data yang sama dan memiliki panjang yang tidak bisa berubah setelah dideklarasi.

Sintaks array baru dengan membuat daftar nilai yang terpisah oleh koma didalam kurung siku \hilite{[]}.
\end{frame}

\begin{frame}[fragile]{Array}
    \begin{minted}{rust}
    // dengan tipe annotation
    let array: [i32; 3] = [1, 2, 3];
    
    // repeat expression
    let mut array: [i32; 3] = [0; 3];
    
    let bulan = ["Januari", "Pebruari", "Maret", "April", "Mei", "Juni", "Juli", "Agustus", "September", "Oktober", "Nopember", "Desember"];
    let jan = bulan[0];
    let dec = bulan[11];
    \end{minted}
\end{frame}

\begin{frame}[fragile]{Slice}
Slice \hilite{(\&[T])} adalah '\textit{view}' untuk koleksi seperti array dengan ukuran yang \emph{dinamis}. Dinamis dalam arti ukuran tidak diketahui saat kompilasi.

Slice menampilkan blok memori dengan representasi sebagai sebuah pointer serta ukuran panjangnya.

\end{frame}

\begin{frame}[fragile]{Slice}
\begin{minted}{rust}
let str_slice: &[&str] = &["one", "two"];

let arr = [1, 2, 3, 4, 5, 6];
let arr_slice: &[i32] = &arr[1..4];

println!("{:?}", arr_slice);  // [2, 3, 4]

assert_eq!(Some(&3), arr_slice.get(1));
// atau
assert_eq!(3, arr_slice[1]);
\end{minted}
\end{frame}

\begin{frame}{String slices}
Tipe dasar \hilite{str} untuk menyimpan kalimat \textit{string}.

Nilai dari strings slices selalu valid UTF-8.
\end{frame}

\begin{frame}[fragile]{String slices}
\begin{minted}{rust}
let hello = "Hello, world!";

// dengan tipe annotation
let hello: &'static str = "Hello, world!";
\end{minted}
\end{frame}

%\begin{frame}{Pointer}
%Tipe dasar pointer adalah penunjuk lokasi memori, ditulis dengan \hilite{*const T} dan \hilite{*mut T}
%\end{frame}
%
%\begin{frame}[fragile]{Pointer}
%    \begin{minted}{rust}
%// Mengambil sebuah referensi
%let my_num: i32 = 10;
%let my_num_ptr: *const i32 = &my_num;
%
%// Referesi mutable
%let mut my_age: i32 = 26;
%let my_age_ptr: *mut i32 = &mut my_age;
%    \end{minted}
%\end{frame}

\begin{frame}{Tipe-tipe Lain}
Rust Standard Library menyediakan banyak tipe antara lain:
\begin{description}
    \item [Vector] \hilite{Vec<T>}, Tipe Array yang dapat bertambah dan dialokasikan di \textit{heap}.
    
    \item [Box] \hilite{Box<T>}, Tipe pointer untuk alokasi data di \textit{heap}.
    
    \item [String] Tipe kalimat \textit{string} yang dapat bertambah dengan encoding UTF-8
    
\end{description}
\end{frame}

\begin{frame}{Konversi Antar Tipe (Casting)}
  \newcolumntype{g}{>{\columncolor{black!20}}m}
  {\small
  \begin{tabular}{|g{1.5cm}|c|c|c|c|}
      \hline
      \rowcolor{black!20}%
      \diagbox[innerwidth=1.5cm]{from}{to} & i32 & u32 & f64 & String \\
      \hline
       \centering i32 & n/a & x as \ttfamily{u32} & x as \ttfamily{f64} & x.to\_string() \\
       \centering u32 & x as \ttfamily{i32} & n/a & x as \ttfamily{f64} & x.to\_string() \\
       \centering f64 & x as \ttfamily{i32} & x as \ttfamily{u32} & n/a & x.to\_string() \\
       \hline
  \end{tabular}
}

    String ke tipe numerik :\\ \mintinline{rust}{x.parse()}

    Secara explicit dengan type annotation  :\\
    \mintinline{rust}{x.parse::<i32>()}
\end{frame}
\begin{frame}{Konversi Antar Tipe String}
\newcolumntype{g}{>{\columncolor{black!20}}m}
\centering
\begin{tabular}{|g{1.5cm}|c|c|c|c|}
    \hline
    \rowcolor{black!20}%
    \diagbox[innerwidth=1.5cm]{from}{to} & String & \&str \\
    \hline
    \centering String & n/a & \&*x \\
    \hline
    \centering \&str & x.to\_string() & n/a  \\
    \hline
\end{tabular}
\end{frame}

\begin{frame}{Kesimpulan}
    \begin{itemize}
        \item Pilih tipe data yang tepat.
        \item Cobalah kasus dengan \textit{test}.
        \item Tangani \textit{error}.
    \end{itemize}
\end{frame}

{
    \usebackgroundtemplate%
    {%
        \tikz[overlay,remember picture] \node[opacity=0.1, xshift=-2.7cm, yshift=2cm, at=(current page.330)] { \includegraphics[width=0.6\paperwidth]{rust-logo-512x512}};}%
    \begin{frame}[plain]
    \Large \textbf{Sekian \& Terima Kasih}
    
\end{frame}
}

\appendix
\begin{frame}{Referensi}

Pustaka

\href{https://doc.rust-lang.org/std/index.html\#primitives}{\beamergotobutton{https://doc.rust-lang.org/std/index.html\#primitives}}\\
\href{https://doc.rust-lang.org/reference/types.html}{\beamergotobutton{https://doc.rust-lang.org/reference/types.html}}
\href{https://doc.rust-lang.org/book/second-edition/ch03-02-data-types.html}{\beamergotobutton{https://doc.rust-lang.org/book/second-edition/ch03-02-data-types.html}}
\href{http://carols10cents.github.io/rust-conversion-reference}{\beamergotobutton{http://carols10cents.github.io/rust-conversion-reference}}\\
\href{https://www.manning.com/books/rust-in-action}{\beamergotobutton{https://www.manning.com/books/rust-in-action}}\\
\href{https://www.doc.ic.ac.uk/~eedwards/compsys/float/}{\beamergotobutton{https://www.doc.ic.ac.uk/\textasciitilde{}eedwards/compsys/float/}}

\end{frame}

\begin{frame}{Referensi}
Kode

\href{https://github.com/rust-lang/rust/blob/master/src/libcore/num}{\beamergotobutton{https://github.com/rust-lang/rust/blob/master/src/libcore/num}}
\href{https://github.com/rust-lang/rust/blob/master/src/test/parse-fail/lex-bad-numeric-literals.rs}{\beamergotobutton{https://github.com/rust-lang/rust/blob/master/src/test/parse-fail/lex-bad-numeric-literals.rs}}

Lain-lain

\href{https://en.wikipedia.org/wiki/Integer\_(computer\_science)}{\beamergotobutton{https://en.wikipedia.org/wiki/Integer\_(computer\_science)}}
\href{https://unicode-table.com/en/\#control-character}{\beamergotobutton{https://unicode-table.com/en/\#control-character}}\\
\href{https://play.rust-lang.org/}{\beamergotobutton{https://play.rust-lang.org/}}\\
\href{https://rust.godbolt.org/}{\beamergotobutton{https://rust.godbolt.org/}}




\end{frame}

\end{document}